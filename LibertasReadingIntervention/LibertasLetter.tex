%!TEX TS-program = xelatex
%!TEX encoding = UTF-8 Unicode
% Awesome CV LaTeX Template for Cover Letter
%
% This template has been downloaded from:
% https://github.com/posquit0/Awesome-CV
%
% Authors:
% Claud D. Park <posquit0.bj@gmail.com>
% Lars Richter <mail@ayeks.de>
%
% Template license:
% CC BY-SA 4.0 (https://creativecommons.org/licenses/by-sa/4.0/)
%


%-------------------------------------------------------------------------------
% CONFIGURATIONS
%-------------------------------------------------------------------------------
% A4 paper size by default, use 'letterpaper' for US letter
\documentclass[11pt, a4paper]{awesome-cv}

% Configure page margins with geometry
\geometry{left=1.4cm, top=.8cm, right=1.4cm, bottom=1.8cm, footskip=.5cm}

% Specify the location of the included fonts
\fontdir[fonts/]

% Color for highlights
% Awesome Colors: awesome-emerald, awesome-skyblue, awesome-red, awesome-pink, awesome-orange
%                 awesome-nephritis, awesome-concrete, awesome-darknight
%\colorlet{awesome}{awesome-red}
% Uncomment if you would like to specify your own color
 \definecolor{awesome}{HTML}{5b9279}

% Colors for text
% Uncomment if you would like to specify your own color
% \definecolor{darktext}{HTML}{414141}
% \definecolor{text}{HTML}{333333}
% \definecolor{graytext}{HTML}{5D5D5D}
% \definecolor{lighttext}{HTML}{999999}

% Set false if you don't want to highlight section with awesome color
\setbool{acvSectionColorHighlight}{true}

% If you would like to change the social information separator from a pipe (|) to something else
\renewcommand{\acvHeaderSocialSep}{\quad\textbar\quad}


%-------------------------------------------------------------------------------
%	PERSONAL INFORMATION
%	Comment any of the lines below if they are not required
%-------------------------------------------------------------------------------
% Available options: circle|rectangle,edge/noedge,left/right
%\photo[circle,edge,left]{./img/bwwenz.png}
\name{Christopher}{Wenz}
%\position{Software Architect{\enskip\cdotp\enskip}Security Expert}
\address{456 Long Plain Road ~·~ Leverett, MA 01054}
\mobile{860 977 1674}
\email{cawenz@gmail.com}
%\homepage{www.posquit0.com}
%\github{cawenz}
\linkedin{chris-a-wenz}
\researchgate{chris\_wenz}
% \gitlab{gitlab-id}
% \stackoverflow{SO-id}{SO-name}
% \twitter{@twit}
% \skype{skype-id}
% \reddit{reddit-id}
% \medium{madium-id}
% \googlescholar{googlescholar-id}{name-to-display}
%% \firstname and \lastname will be used
% \googlescholar{googlescholar-id}{}
% \extrainfo{extra informations}

%\quote{``Be the change that you want to see in the world."}


%-------------------------------------------------------------------------------
%	LETTER INFORMATION
%	All of the below lines must be filled out
%-------------------------------------------------------------------------------
% The company being applied to
\recipient
 {Mr. Modesto Montero}
 {146 Chestnut Street\\Springfield, MA 01103}
% The date on the letter, default is the date of compilation
\letterdate{\today}
% The title of the letter
\lettertitle{Candidate for Reading Intervention Director}
% How the letter is opened
\letteropening{\\\\Dear Mr. Montero,}
% How the letter is closed
\letterclosing{Sincerely,}
% Any enclosures with the letter
%\letterenclosure[Attached]{Curriculum Vitae}


%-------------------------------------------------------------------------------
\begin{document}

% Print the header with above personal informations
% Give optional argument to change alignment(C: center, L: left, R: right)
\makecvheader[C]

% Print the footer with 3 arguments(<left>, <center>, <right>)
% Leave any of these blank if they are not needed
\makecvfooter
  {\today}
  {Chris Wenz~~~·~~~Cover Letter}
  {Page \thepage\ of 2}
% Print the title with above letter informations
\makelettertitle

%-------------------------------------------------------------------------------
%	LETTER CONTENT
%-------------------------------------------------------------------------------
\begin{cvletter}

I am incredibly excited by the opportunity to join you and your team at Libertas Academy. Although much of my recent experience has been in higher education, I believe that my expertise in literacy assessment, instruction and teacher education would make me an asset to your program and the students it serves.

Since leaving my position as Research Scientist at Landmark College in June, I have been teaching as an Adjunct Professor at the University of Connecticut and working to complete my PhD. I have also been providing full-time care for my 5 year-old son and volunteering in support of a much-needed renovation of his non-profit preschool: Willow Blossom Learning Center. In my role as Board President, I have collaborated with the Center Director and staff to upgrade the building's HVAC systems, re-design classrooms, and secure the funding needed for these renovations. I have also volunteered hundreds of hours to complete the carpentry and painting necessary to create a safer and more functional space for the Program. This work has been particularly important to me because Willow Blossom is one of the few providers in the area that accepts state childcare vouchers. The work we have done at the Willow Blossom this year will ensure that there is quality, affordable childcare available in Franklin county and Willow Blossom can continue to serve a diverse group of children and families. Now that Willow Blossom has re-opened, I am currently looking for a position that allows my family to stay in the Pioneer Valley long-term and affords me the opportunity to apply my skills as an educator. 

In August, I will complete my PhD in Curriculum and Instruction; my graduate studies have built on my classroom experience as a Special Educator and I have focused on adolescent literacy instruction, assessment and intervention. Much of my scholarly work has been focused on "disciplinary literacy" (DL): an approach to instruction that focuses on developing students' literacies by engaging them in the literate practices of disciplinary experts. My approach to literacy instruction is based on a belief that all students, regardless of their reading abilities, can and do extend their literacies through authentic engagement with disciplinary content; I believe this approach is an excellent fit for the rigorous academic program you are building at Libertas Academy. As a Graduate Assistant and Adjunct Professor at the University of Connecicut, I have applied these ideas to numerous courses on literacy instruction for pre-service teacher candidates and students in the Literacy Specialist Program. I believe that my experience as a teacher educator would enable me to help \emph{all} teachers at Libertas Academy deliver powerful and engaging literacy instruction for \emph{all} students. 

In addition my experience as a teacher educator, I have expertise in reading assessment and intervention, and designing instruction to support students with reading differences and disabilities. I have extensive experience with a broad range of assessment tools, and have analyzed individual and school-level data to develop intervention plans and design small-group courses for students requiring additional support. In particular, as a Research Assistant at UConn, I supported a long-term literacy initiative in the Windsor (CT) Public Schools that included a re-design of reading intervention coures. Over the past 18 months, I have also been working with the Reading Recovery Council of North America to analyze 5 years of data (n > 100,000) in order to identify instructional changes that could make their intervention more effective for a broader range of students, including emergent bilingual students. I am confident that my experience and expertise have prepared me well to create systems and supports that help every student at Libertas Academy grow as readers and writers. 

I am an experienced and committed educator and teacher educator, and I would greatly appreciate the opportunity to discuss my qualifications with you in an interview. Further, I would love to learn more about the plans and vision for literacy instruction at Libertas. Thank you for your consideration.  
\\\\
Best,
\end{cvletter}
%-------------------------------------------------------------------------------
% Print the signature and enclosures with above letter informations
%\makeletterclosing

\includegraphics[width=4cm, height=0.8cm]{./img/sig.png}
\end{document}
