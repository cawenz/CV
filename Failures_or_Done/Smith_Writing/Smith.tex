%!TEX TS-program = xelatex
%!TEX encoding = UTF-8 Unicode
% Awesome CV LaTeX Template for Cover Letter
%
% This template has been downloaded from:
% https://github.com/posquit0/Awesome-CV
%
% Authors:
% Claud D. Park <posquit0.bj@gmail.com>
% Lars Richter <mail@ayeks.de>
%
% Template license:
% CC BY-SA 4.0 (https://creativecommons.org/licenses/by-sa/4.0/)
%


%-------------------------------------------------------------------------------
% CONFIGURATIONS
%-------------------------------------------------------------------------------
% A4 paper size by default, use 'letterpaper' for US letter
\documentclass[11pt, a4paper]{awesome-cv}

% Configure page margins with geometry
\geometry{left=1.4cm, top=.8cm, right=1.4cm, bottom=1.8cm, footskip=.5cm}

% Specify the location of the included fonts
\fontdir[fonts/]

% Color for highlights
% Awesome Colors: awesome-emerald, awesome-skyblue, awesome-red, awesome-pink, awesome-orange
%                 awesome-nephritis, awesome-concrete, awesome-darknight
%\colorlet{awesome}{awesome-red}
% Uncomment if you would like to specify your own color
 \definecolor{awesome}{HTML}{5b9279}

% Colors for text
% Uncomment if you would like to specify your own color
% \definecolor{darktext}{HTML}{414141}
% \definecolor{text}{HTML}{333333}
% \definecolor{graytext}{HTML}{5D5D5D}
% \definecolor{lighttext}{HTML}{999999}

% Set false if you don't want to highlight section with awesome color
\setbool{acvSectionColorHighlight}{true}

% If you would like to change the social information separator from a pipe (|) to something else
\renewcommand{\acvHeaderSocialSep}{\quad\textbar\quad}


%-------------------------------------------------------------------------------
%	PERSONAL INFORMATION
%	Comment any of the lines below if they are not required
%-------------------------------------------------------------------------------
% Available options: circle|rectangle,edge/noedge,left/right
%\photo[circle,edge,left]{./img/bwwenz.png}
\name{Christopher}{Wenz}
%\position{Software Architect{\enskip\cdotp\enskip}Security Expert}
\address{456 Long Plain Road ~·~ Leverett, MA 01054}
\mobile{860 977 1674}
\email{cawenz@gmail.com}
%\homepage{www.posquit0.com}
%\github{cawenz}
\linkedin{chris-a-wenz}
\researchgate{chris\_wenz}
% \gitlab{gitlab-id}
% \stackoverflow{SO-id}{SO-name}
% \twitter{@twit}
% \skype{skype-id}
% \reddit{reddit-id}
% \medium{madium-id}
% \googlescholar{googlescholar-id}{name-to-display}
%% \firstname and \lastname will be used
% \googlescholar{googlescholar-id}{}
% \extrainfo{extra informations}

%\quote{``Be the change that you want to see in the world."}


%-------------------------------------------------------------------------------
%	LETTER INFORMATION
%	All of the below lines must be filled out
%-------------------------------------------------------------------------------
% The company being applied to
\recipient
 {Dr. Julio Alves}
 {10 Elm Street\\Northampton, MA 01063}
% The date on the letter, default is the date of compilation
\letterdate{\today}
% The title of the letter
\lettertitle{Candidate for Writing Instructor and Writing Studies Specialist}
% How the letter is opened
\letteropening{\\\\Dear Dr. Alves and members of the hiring committee,}
% How the letter is closed
\letterclosing{Sincerely,}
% Any enclosures with the letter
%\letterenclosure[Attached]{Curriculum Vitae}


%-------------------------------------------------------------------------------
\begin{document}

% Print the header with above personal informations
% Give optional argument to change alignment(C: center, L: left, R: right)
\makecvheader[C]

% Print the footer with 3 arguments(<left>, <center>, <right>)
% Leave any of these blank if they are not needed
\makecvfooter
  {\today}
  {Chris Wenz~~~·~~~Cover Letter}
  {Page 1 of 1}

% Print the title with above letter informations
\makelettertitle

%-------------------------------------------------------------------------------
%	LETTER CONTENT
%-------------------------------------------------------------------------------
\begin{cvletter}

In August, I will complete my PhD in Curriculum and Instruction; much of my scholarly work and college teaching has been focused on literacy instruction across the disciplines. In courses I've designed at the graduate and undergraduate level, I have helped K-12 and college instructors design "authentic" writing tasks that align with the purposes, formats and audiences of academic disciplines. In my own K-12 teaching, I have provided  students with opportunities to write for real audiences in a variety of formats; for example, I am currently revising a manuscript submitted to \emph{The History Teacher} which centers on my experience teaching high school students to create or edit wikipedia articles about local public art (as a means for learning local history as well as historical thinking and communication). I would love the opportunity to apply my knowledge and experience teaching writing (and writing pedagogy) to the Jacobsen Center's efforts to support instruction across the college. 

In addition to my expertise in literacy pedagogy, I have extensive experience working with neurodivergent students in both K-12 and college settings. As a Graduate Writing Tutor at UConn, I emphasized my own learning challenges and my experience teaching students with learning differences in my public bio published on the Writing Center's website. As a result, it was common for students with learning differences and, in particular, writing anxiety to book their appointments with me. Often, my work with these students deviated from the typical structure of a tutoring session: the typical was not going to work for them. These experiences and my years as a Special Educator have made clear to me that there is no one method or pedagogical approach that can account for the diversity of learners that exist in every setting; instead, good instruction requires compassion, understanding and a willingnes to respond to the range of needs that define every group of learners. In addition to providing quality support to Smith students (regardless of their learning needs) I believe I could help Smith faculty support the writing development of \emph{all} students. 

I believe that my experience as a researcher, writing tutor, educator and teacher educator would make me an asset to Smith students, faculty and the Jacobsen Center. I would greatly appreciate the opportunity to discuss my qualifications with you in an interview; thank you for your consideration. 

\\\\
Best,
\end{cvletter}


%-------------------------------------------------------------------------------
% Print the signature and enclosures with above letter informations
%\makeletterclosing

\includegraphics[width=4cm, height=0.8cm]{./img/sig.png}
\end{document}
